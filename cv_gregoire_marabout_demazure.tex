%%%%%%%%%%%%%%%%%
% This is an sample CV template created using altacv.cls
% (v1.6, 21 May 2021) written by LianTze Lim (liantze@gmail.com). Now compiles with pdfLaTeX, XeLaTeX and LuaLaTeX.
%
%% It may be distributed and/or modified under the
%% conditions of the LaTeX Project Public License, either version 1.3
%% of this license or (at your option) any later version.
%% The latest version of this license is in
%%    http://www.latex-project.org/lppl.txt
%% and version 1.3 or later is part of all distributions of LaTeX
%% version 2003/12/01 or later.
%%%%%%%%%%%%%%%%

%% Use the "normalphoto" option if you want a normal photo instead of cropped to a circle
% \documentclass[10pt,a4paper,normalphoto]{altacv}

\documentclass[10pt,a4paper,ragged2e,withhyper]{altacv}
%% AltaCV uses the fontawesome5 and packages.
%% See http://texdoc.net/pkg/fontawesome5 for full list of symbols.

% Change the page layout if you need to
\geometry{left=1.25cm,right=1.25cm,top=1.5cm,bottom=1.5cm,columnsep=1.2cm}

% The paracol package lets you typeset columns of text in parallel
\usepackage{paracol}

% Change the font if you want to, depending on whether
% you're using pdflatex or xelatex/lualatex
\ifxetexorluatex
% If using xelatex or lualatex:
\setmainfont{Roboto Slab}
\setsansfont{Lato}
\renewcommand{\familydefault}{\sfdefault}
\else
% If using pdflatex:
\usepackage[rm]{roboto}
\usepackage[defaultsans]{lato}
% \usepackage{sourcesanspro}
\renewcommand{\familydefault}{\sfdefault}
\fi

% Change the colours if you want to
\definecolor{SlateGrey}{HTML}{2E2E2E}
\definecolor{LightGrey}{HTML}{666666}
\definecolor{DarkPastelRed}{HTML}{450808}
\definecolor{PastelRed}{HTML}{8F0D0D}
\definecolor{GoldenEarth}{HTML}{E7D192}
\colorlet{name}{black}
\colorlet{tagline}{PastelRed}
\colorlet{heading}{DarkPastelRed}
\colorlet{headingrule}{GoldenEarth}
\colorlet{subheading}{PastelRed}
\colorlet{accent}{PastelRed}
\colorlet{emphasis}{SlateGrey}
\colorlet{body}{LightGrey}

% Change some fonts, if necessary
\renewcommand{\namefont}{\Huge\rmfamily\bfseries}
\renewcommand{\personalinfofont}{\footnotesize}
\renewcommand{\cvsectionfont}{\LARGE\rmfamily\bfseries}
\renewcommand{\cvsubsectionfont}{\large\bfseries}

% Change the bullets for itemize and rating marker
% for \cvskill if you want to
\renewcommand{\itemmarker}{{\small\textbullet}}
\renewcommand{\ratingmarker}{\faCircle}
%% Use (and optionally edit if necessary) this .tex if you
%% want to use an author-year reference style like APA(6)
% \input{pubs-authoryear}


\begin{document}
	\name{Grégoire Marabout-Demazure}
	\tagline{CTO | Senior Software Engineer}

	\photoR{2.8cm}{greg.jpg}
	
	\personalinfo{%

		\email{gmarabout@gmail.com}
		\phone{+33 6 60 65 64 61}
		\mailaddress{10, rue George Sand 34920 Le Crès}
		\location{Montpellier Area, France}
		\twitter{@gmarabout}
		\linkedin{gmarabout}
		\github{gmarabout}
		%% You can add your own arbitrary detail with
		%% \printinfo{symbol}{detail}[optional hyperlink prefix]
		% \printinfo{\faPaw}{Hey ho!}[https://example.com/]
		%% Or you can declare your own field with
		%% \NewInfoFiled{fieldname}{symbol}[optional hyperlink prefix] and use it:
		% \NewInfoField{gitlab}{\faGitlab}[https://gitlab.com/]
		% \gitlab{your_id}
		%%
		%% For services and platforms like Mastodon where there isn't a
		%% straightforward relation between the user ID/nickname and the hyperlink,
		%% you can use \printinfo directly e.g.
		% \printinfo{\faMastodon}{@username@instace}[https://instance.url/@username]
		%% But if you absolutely want to create new dedicated info fields for
		%% such platforms, then use \NewInfoField* with a star:
		% \NewInfoField*{mastodon}{\faMastodon}
		%% then you can use \mastodon, with TWO arguments where the 2nd argument is
		%% the full hyperlink.
		% \mastodon{@username@instance}{https://instance.url/@username}
	}
	
	\makecvheader
	%% Depending on your tastes, you may want to make fonts of itemize environments slightly smaller
	% \AtBeginEnvironment{itemize}{\small}
	
	%% Set the left/right column width ratio to 6:4.
	\columnratio{0.6}
	
	% Start a 2-column paracol. Both the left and right columns will automatically
	% break across pages if things get too long.
	\begin{paracol}{2}
		\cvsection{Experience}
				\cvevent{Independant Software Engineer and Architect}{Liberscale}{Oct 2022 -- now}{Montpellier, France}
	\begin{itemize}
		\item Cloud \& micro-services architecture
		\item Cybersecurity, Data \& AI applications development
	\end{itemize}
	
		
		\cvtag{Python}
		\cvtag{Open-source}
		
		\divider
		
		
		\cvevent{Engineering Lead}{Datadog}{Sept 2020 -- Oct 2022}{Montpellier, France}
		\begin{itemize}
			\item Build Datadog  network devices monitoring platform
			\item Build and Lead development team (5 persons)
		\end{itemize}
	
		\cvtag{Leadership}
		\cvtag{Golang} 
		\cvtag{Kubernetes}
		\cvtag{Open-source}
		
		\divider
		
		\cvevent{CTO \& Co-founder}{DecisionBrain}{Oct 2013 -- Sept 2021}{Montpellier, France}
		\begin{itemize}
			\item Drive DecisionBrain’s technology strategy
			\item Design solutions and products integrating \textit{OR} technology (CPLEX)
		\end{itemize}	
		\cvtag{Leadership}
		\cvtag{Scala} 
		\cvtag{TypeScript}
		\cvtag{Kubernetes}
			
		\divider
		
		\cvevent{Software Architect}{Qualtera}{Jan 2011 -- Sept 2013}{Montpellier, France}
		\begin{itemize}
			\item Build company real-time analytics Cloud platform 
		 for the semi-conductor industry
		\end{itemize}	
		\cvtag{Java}
		\cvtag{GWT} 
		\cvtag{ElasticSearch}
		\cvtag{Apache Camel}
		
		\divider
			
		\cvevent{Software Architect}{Self-Employed}{Jan 2008 -- Jan 2011}{Paris, France}
		\begin{itemize}
			\item Build a game data analysis platform for Mimesis-Republic
			\item Open-source a data analysis tool for Hadoop (cascading.jruby)
		\end{itemize}
		\cvtag{Hadoop}
		\cvtag{Ruby} 
		\cvtag{Open-source}
		
			\divider
		
		\cvevent{Support Engineer, Software Developer,  Architect}{ILOG}{Jan 2000 -- Sept 2008}{Paris, France - Mountain View, USA}
		\begin{itemize}
			\item Build ILOG Business Rules visual editors and modeling tools
			\item Build ILOG Ruleflow editor
		\end{itemize}	
		\cvtag{Java}
		\cvtag{Eclipse}
		
		\medskip

	
		%% Switch to the right column. This will now automatically move to the second
		%% page if the content is too long.
		\switchcolumn
		
		\cvsection{My Life Philosophy}
		
		\begin{quote}
			``The future depends on what you do today.''
		\end{quote}
		
		\cvsection{Most Proud of}
		
		\cvachievement{\faFlask}{Technology}{Built Datadog next generation monitoring tool for network equipments}
		
		\divider
			
		\cvachievement{\faChartLine}{Company's growth}{Extended DecisionBrain’s business by building its Cloud and “industry-specific solutions” strategies}
		
		\divider

		
		\cvachievement{\faPeopleCarry}{Team Builder}{Built teams of highly skilled engineers, with high diversity: 30\% females, 10 different nationalities, different profiles}
		
		\cvsection{Strengths}
		
		\cvtag{Hard-working}
		\cvtag{Motivator}
		\cvtag{Leadership}
		
		\divider\smallskip
		
		\cvtag{Golang}
		\cvtag{Java}
		\cvtag{Python}
		\cvtag{Kubernetes}
		
		\cvsection{Languages}
		
		\cvskill{French}{5}
		\divider
		
		\cvskill{English}{4}
		\divider
		
		\cvskill{Russian}{1} %% Supports X.5 values.
		
		%% Yeah I didn't spend too much time making all the
		%% spacing consistent... sorry. Use \smallskip, \medskip,
		%% \bigskip, \vspace etc to make adjustments.
		\medskip
		
		\cvsection{Education}
		
		\cvevent{B.S. in Computer Science}{Paris VI University}{Sept 1996 -- Aug 1998}{}
		
		\divider
		
		\cvevent{Associate degree in Physics}{Paris VI University}{Sept 1994 -- Aug 1996}{}
		
		
	\end{paracol}
	
	
\end{document}
